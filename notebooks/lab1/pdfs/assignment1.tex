\documentclass[12pt]{article}

\input{../../../worksheets/latex_env}
\rhead{Assignment One (version: \today)}



\begin{document}

\section*{Assignment Set for Laboratory 1}
{\it ATSC 409: Hand-in your answer to question 1.\\
EOSC 511: Hand-in your answer to question 2}\\[12pt]

\noindent{\bf All questions should be done by hand (not by computer)
  and show your steps. Upload your solutions to CANVAS}\\[12pt]

\begin{enumerate}
\item The equation
\begin{equation}
\dydt + c \dydx = 0,\ y = \cos(x) \Mathat t=0,\ \dydt = c \sin(x) \Mathat t=0
\end{equation}
has a solution $y=\cos(x-ct)$.
\begin{enumerate}
\item Expand both derivatives as centred differences. {\it Be very
    clear about indexing in $x$ and $t$ separately.  Notation is up to
    you as long as it is clear, but I suggest, for example $y (x=dx, t=0)$}
\item Show that the algebraic solution is an exact solution of the
  difference formula if we choose $\Delta x = c \Delta t$.  {\it
    Remember for proofs (or shows) like this question, it is important
    not to assume what you are trying to prove.  Work the
    left-hand-side and right-hand-side separately and show they are equal}
\end{enumerate}
\item Given
\begin{equation}
\dydt = -\alpha y,\ y = 1 \Mathat t=0
\end{equation}
\begin{enumerate}
\item Show that the forward Euler method gets a smaller answer than
  the backward Euler method for all $t > 0$, provided that $0 <
  \alpha^2 \Delta t^2 < 1$.
\item Solve the equation analytically.
\item Show that the forward Euler always under-estimates the answer provided that $\alpha \Delta t < 1 \Mathand \alpha \Delta t \ne 0$.
\end{enumerate}\end{enumerate}

\end{document}

%%% Local Variables:
%%% mode: latex
%%% TeX-master: t
%%% End:
